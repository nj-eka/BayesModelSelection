\documentclass[12pt, twoside]{article}
\usepackage[utf8]{inputenc}
\usepackage[english,russian]{babel}
\newcommand{\hdir}{.}

\usepackage{graphicx}
\usepackage{caption}
\usepackage{amssymb}
\usepackage{mathrsfs}
\usepackage{euscript}
\usepackage{upgreek}
\usepackage{array}
\usepackage{theorem}
\usepackage{graphicx}
\usepackage{subfig}
\usepackage{caption}
\usepackage{color}
\usepackage{url}
\usepackage{amsmath}

\usepackage[left=2cm, right=2cm, top=3cm, bottom=3cm, bindingoffset=0cm]{geometry}

\newcommand{\Pb}{\mathcal{P}}

\setcounter{secnumdepth}{-1}

\begin{document} 

\title{Задание 2 по курсу "Байесовский выбор модели"}
\date{}
\maketitle
\begin{center}
{\LARGE Гадаев Тамаз (группа 574),\par
Грабовой Андрей (группа 574),\par
Малиновский Григорий (группа 574),\par
Рогозина Анна (группа 574), \par
Шульгин Егор (группа 574), \par
}
\end{center}
\begin{table}[h!]
\begin{center}
\begin{tabular}{|c|c|c|}
\hline
	Метод&Весс на класс 1& Весс на класс 2\\
	\hline
	AUC &  0.5 & 0.5 \\
	\hline
	NUM & 0.5 & 0.5\\
	\hline
	ASY1& 0.95 & 0.05 \\
	\hline
	ASY2& 0.60 & 0.4 \\
\hline
\end{tabular}
\end{center}
\caption{Таблица важности(новая)}
\label{table1}
\end{table}
\section{Исправления(общие)}
1. Исправлена таблица важности(для ASY1 и для ASY2), новая важность объекта показана в табл.~\ref{table1}
2. Добавлен тест шапиро, для обоснования использования наивного байесового классификатора(о том что прищзнаки распределены нормально). Добавлено нахождение попарной корреляции признаков.
3. Изменена система oversamplinga. Теперь мы обучаемся на произвольных сбалансированных подвыборках и после этого смешиваем обученные модели(голосование моделей устраиваем). Этот метод имеет название Метод случайных сбалансированных подмножеств, который описан в работе~\cite{work1}

\section{Задача 1}
\section{Задача 2}
\section{Задача 3}
\section{Задача 4}
\section{Задача 5}
\section{Задача 6}
\section{Задача 7}
\section{Задача 8}
\section{Задача 9}
\section{Задача 10}
\section{Задача 11}
\section{Задача 12}
\section{Задача 13}
\section{Задача 14}

\begin{thebibliography}{1}
\bibitem{work1}
    \emph{Никулин В. Н., Канищев И. С., Багаев И. В.}
    МЕТОДЫ БАЛАНСИРОВКИ И НОРМАЛИЗАЦИИ ДАННЫХ ДЛЯ УЛУЧШЕНИЯ КАЧЕСТВА КЛАССИФИКАЦИИ~//
   Компьютерные инструменты в образовании, No 3, 2016.~--- p.\,16–24.
    \url{https://link.springer.com/article/10.1134/S000511791808009X}.
\end{thebibliography}
\end{document} 